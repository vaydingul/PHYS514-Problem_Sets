% DO NOT MODIFY THIS SECTION unless you know what you're doing!z
\documentclass[letterpaper,12pt]{article}

\usepackage{tabularx} % extra features for tabular environment
\usepackage{amsmath}  % improve math presentation
\usepackage{amssymb}
\usepackage{multirow}
\usepackage{xcolor}
\usepackage{gensymb}
\usepackage{appendix}
\usepackage{gensymb}
\usepackage{float}
\usepackage{listings}
\usepackage[export]{adjustbox}
\usepackage{graphicx} % takes care of graphic including machinery
\usepackage[margin=1in,letterpaper]{geometry} % decreases margins
\usepackage{cite} % takes care of citations
\usepackage[final]{hyperref} % adds hyper links inside the generated pdf file
\hypersetup{
    colorlinks=false,       % false: boxed links; true: colored links
    linkcolor=blue,        % color of internal links
    citecolor=blue,        % color of links to bibliography
    filecolor=magenta,     % color of file links
    urlcolor=blue         
}
%++++++++++++++++++++++++++++++++++++++++++++++++++++++++++++++++++++++++++++++++



%++++++++++++++++++++++++++++++++++++++++++++++++++++++++++++++++++++++++++++++++
% Start modifying the labwork number, your team number and the name and METU id
% of your group members.
\newcommand{\reporttitle}{Problem Set 1}
\newcommand{\reportauthor}{ Volkan Aydıngül (Id: 0075359 )\\
                            }
                            % If any teammate does not help to write this report,
                            % you may not write his/her name here.
%++++++++++++++++++++++++++++++++++++++++++++++++++++++++++++++++++++++++++++++++



%++++++++++++++++++++++++++++++++++++++++++++++++++++++++++++++++++++++++++++++++
% DO NOT MODIFY THIS SECTION
\begin{document}
\begin{titlepage}
\newcommand{\HRule}{\rule{\linewidth}{0.5mm}}
\begin{center} % Center remainder of the page
%	LOGO SECTION
\includegraphics[width = 8cm]{figures/koc_logo.png}

%	HEADING SECTIONS
\textsc{\Large PHYS 514 - Computational Physics}\\[1.5cm] 
%	TITLE SECTION
\HRule \\[0.6cm]
{ \huge \bfseries \reporttitle}\\ % Title of your document
\HRule \\[1.5cm]
\end{center}
\vspace{2cm}
%	AUTHOR SECTION
\begin{flushleft} \large
\textit{Author:}\\
\reportauthor% Your name
\end{flushleft}
\vspace{2cm}
\makeatletter
Date: \@date 
\vfill % Fill the rest of the page with whitespace
\makeatother
\end{titlepage}
%++++++++++++++++++++++++++++++++++++++++++++++++++++++++++++++++++++++++++++++++




\tableofcontents
\newpage





%\begin{figure}[H] 
%   \centering \includegraphics[width=\columnwidth]{figures/figure.png}           
%                \caption{Caption}                
%                   \label{fig:label}
%   \end{figure}

\section{Problem I}
\subsection{Calculation of $1 - \frac{\sqrt{1+x^2}}{1+\frac{1}{2}x^2}$}
\paragraph{}In this problem, we are asked to calcute following function:
\begin{equation*}
  f(x) =  1 - \frac{\sqrt{1+x^2}}{1+\frac{1}{2}x^2}
\end{equation*}
\paragraph{} In addition to the calculation, also the minimum precision level of $10^{-8}$ is required. Since we are working on $64 \; bit$ machines, we have always absolute error of $10^{-16}$. First, we need to find the $x$ value which corresponds to border of precision, namely $10^{-8}$. To do that, one can conduct following relation:
\begin{equation*}
   Relative \; Error = \frac{Absolute \; Error}{f(x)}
\end{equation*}

\begin{equation*}
   10^{-8}  = \frac{10^{-16}}{f(x)}
\end{equation*}

Above equation can be solved by \textit{Wolfram Mathematica}, and approximate solution can be found as:
\begin{equation*}
   x \approx 10^{-2}
\end{equation*}

\paragraph{} Now, we know that if the $x$ value drops below $10^{-2}$, we start to lose precision that we wanted. The basic idea behind this situation is the fact that when $x$ is very low, quatioent part starts to approximate $1$ and there happens a loss of signifance due to the $(1-1)$. To deal with this situation, we can expand power series. With help of \textit{Wolfram Mathematica}, one can obtain following power series for $f(x)$.
\begin{equation*}
   f(x) = 0.125x^4 - 0.125x^6 + 0.101563x^8-0.078125 x^{10} + O(x^{12})
\end{equation*}

When we input of $x$ around order of magnitude of $10^{-2}$, we obtain approximately following values:
\begin{equation*}
   f(x) \approxeq 10^{-8} + 10^{-12} + 10^{-16} + 10^{-20} + \dots 
\end{equation*}
We know that criteria to truncate the series for a given precision is:
\begin{equation*}
   \left\lvert \frac{a_{k+1}x_{k+1}}{a_1x_1} \right\rvert < Precision
\end{equation*}
Satisfying above inequality, we deduce that the first three term is sufficient to obtain precision of $10^{-8}$, when the $x$ is lower than $10^{-2}$. Finally, we can write our function in a piece-wise manner as following:

\[ f(x) = 
\begin{cases}
   1 - \frac{\sqrt{1+x^2}}{1+\frac{1}{2}x^2} & x\geq 10^{-2} \\
   0.125x^4 - 0.125x^6 + 0.101563x^8 &  x < 10^{-2} \\
      
\end{cases} 
 \]

 The \textit{loglog} plot that represents the $f(x)$ can be seen in Figure \ref{fig:loglogb}.

 begin{figure}[H] 
   \centering \includegraphics[width=\columnwidth]{figures/figure.png}           
                \caption{Caption}                
                   \label{fig:label}
   \end{figure}


 \subsection{Calculation of $\cot x^2 - \frac{1}{x^2}$}
 \paragraph{}Again, in this problem, we will exactly apply above procedure. In this case, our function is:
 \begin{equation*}
   f(x) =  \cot x^2 - \frac{1}{x^2}
 \end{equation*}
 \paragraph{} First, we need to find the $x$ that corresponds to loss of precision of $10^{-8}$. Again, by using \textit{Wolfram Mathematica}, we find that:
 \begin{equation*}
    Relative \; Error = \frac{Absolute \; Error}{f(x)}
 \end{equation*}
 
 \begin{equation*}
    10^{-8}  = \frac{10^{-16}}{f(x)}
 \end{equation*}
 
 \begin{equation*}
    x \approx 10^{-4}
 \end{equation*}
 
 \paragraph{} Let's expand power series with help of \textit{Wolfram Mathematica}:
 \begin{equation*}
    f(x) =-\frac{x^2}{3}-\frac{x^6}{45}-\frac{2 x^{10}}{945}-\frac{x^{14}}{4725}-\frac{2 x^{18}}{93555}+O\left(x^{21}\right)
 \end{equation*}
 
 When we input of $x$ around order of magnitude of $10^{-4}$, we obtain approximately following values:
 \begin{equation*}
    f(x) \approxeq 10^{-8} + 10^{-24} + 10^{-40} + 10^{-56} + 10^{-80} + \dots 
 \end{equation*}
 We know that criteria to truncate the series for a given precision is:
 \begin{equation*}
    \left\lvert \frac{a_{k+1}x_{k+1}}{a_1x_1} \right\rvert < Precision
 \end{equation*}
 Satisfying above inequality, we deduce that the first term is sufficient to obtain precision of $10^{-8}$, when the $x$ is lower than $10^{-4}$. Finally, we can write our function in a piece-wise manner as following:
 
 \[ f(x) = 
 \begin{cases}
   \cot x^2 - \frac{1}{x^2} & x\geq 10^{-4} \\
   -\frac{x^2}{3} &  x < 10^{-4} \\
       
 \end{cases} 
  \]
 




\section{Problem II - Roots of Quadratic Equation}
\paragraph{}Let's illustrate the quadratic equation, 
\begin{equation*}
   ax^2 + bx + c = 0
\end{equation*}
In this case, the roots of the above equation can be defined as follows:
\begin{equation*}
   x_{1,2} = \frac{-b \pm \sqrt{b^2 - 4ac}}{2a}
\end{equation*}
\paragraph{}However, when we use above form for root finding, some problems may occur. One of these problems is that when the difference between the terms $b^2$ and $4ac$ is very large, the loss of significance may take place. For example, when $c$ is very small, depending on the sign of the $b$, the aforamentioned loss of significance can reduce the accuracy dramatically. In essence, the more stable algorithm which eliminates the usage of substraction is needed.
\subsection{Alternative Form for Root Finding}
\paragraph{}In alternative method, the purpose is to eliminate the need for substraction, which is the main source of the loss of significance. When $a$ or $c$ is very small compared to the $b^2$, the square term is very likely to result in very close value to $b$. In this case, ultimately, we need to process following calculation:
\begin{equation*}
   -b + b^*
\end{equation*}
where $b^*$ stands for \textit{very close number to $b$}. To get rid of the catastrophic cancellation that may occur due to the above calculation, we can use below formula to calculate the one of the roots.

\begin{equation*}
   x_{1} = \frac{-b - sgn(b)\sqrt{b^2 - 4ac}}{2a}
\end{equation*}

By doing that, we can only encounter with either summation of two positive numbers or summation of two negative numbers. Finally, the second root can be calculated as follows:
\begin{equation*}
   x_{2} = \frac{c}{a x_1}
\end{equation*}

The formula to find second root is simply derived from the \textit{multiplication of the roots} relation:
\begin{equation*}
   x_1x_{2} = \frac{c}{a}
\end{equation*}

\section{Problem III}
\section{Problem IV}

\end{document}

              


